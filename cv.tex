%%%%%%%%%%%%%%%%%%%%%%%%%%%%%%%%%%%%%%%%%
% Friggeri Resume/CV
% XeLaTeX Template
% Version 1.2 (3/5/15)
%
% This template has been downloaded from:
% http://www.LaTeXTemplates.com
%
% Original author:
% Adrien Friggeri (adrien@friggeri.net)
% https://github.com/afriggeri/CV
%
% License:
% CC BY-NC-SA 3.0 (http://creativecommons.org/licenses/by-nc-sa/3.0/)
%
% Important notes:
% This template needs to be compiled with XeLaTeX and the bibliography, if used,
% needs to be compiled with biber rather than bibtex.
%
%%%%%%%%%%%%%%%%%%%%%%%%%%%%%%%%%%%%%%%%%

\documentclass[]{scoggins-cv} % Add 'print' as an option into the square bracket to remove colors from this template for printing

\usepackage{fontspec}
\newfontfamily{\FA}[Path = fonts/]{fontawesome-webfont}

\usepackage{ragged2e}

%\usepackage{marginnote}
\usepackage{geometry}
\geometry{
  verbose,
  tmargin=2cm,
  bmargin=2cm,
  lmargin=2cm,
  rmargin=2cm
}
\def\faEnvelope{{\FA\symbol{"F0E0}}}
\def\faGithub{{\FA\symbol{"F09B}}}
\def\faDesktop{{\FA\symbol{"F108}}}
\def\faPhone{{\FA\symbol{"F095}}}
\def\faHandPointerO{{\FA\symbol{"F25A}}}
\def\faMapMarker{{\FA\symbol{"F041}}}
\def\faVideoCamera{{\FA\symbol{"F03D}}}
\def\faCloudDownload{{\FA\symbol{"F0ED}}}

\input{citationskip}

\begin{document}

%\header
\section{Matthew T. Scoggins}
%----------------------------------------------------------------------------------------
%	Contact
%----------------------------------------------------------------------------------------
%\iffalse
%\aside{Contact}{-.5035}{
    %
%    \vspace*{-0.1cm}
    %
    %
%    \href{https://github.com/mscoggs}{{\addfontfeature{Color=linkblue}github.com/mscoggs}\ \faGithub}\\[0.5em]
    %
%    \href{mscoggs.github.io}{{\addfontfeature{Color=linkblue}mscoggs.github.io}\ \faHandPointerO}\\[0.5em]
    %
 %   \href{tel:360-325-3398}{{\addfontfeature{Color=linkblue}+1 (360) 325 3398}\ \faPhone}\\[0.5em]
  %  \href{mailto:mts2188@columbia.edu}{{\addfontfeature{Color=linkblue}mts2188@columbia.edu}\ \faEnvelope}\\[.01em]
    %
    %\href{https://www.astro.columbia.edu/content/matthew-scoggins}{{\addfontfeature{Color=linkblue}\small Astronomy Department }\\
    %    \href{https://www.astro.columbia.edu/content/matthew-scoggins}{{\addfontfeature{Color=linkblue}\small Columbia University}}\ \faMapMarker}
    %
%}
%\fi
%----------------------------------------------------------------------------------------
%	EDUCATION
%----------------------------------------------------------------------------------------
\vspace{-0.5cm}
\section{Education}
\vspace{-0.3cm}

\begin{entrylist}

    %------------------------------------------------

	\vspace{-0.3cm}
    \entry
    {2021--}
    {PhD {\normalfont Astrophysics}}
    {Columbia University, New York, NY}
	{}


    %------------------------------------------------

    \entry
    {2015--2020}
    {BS {\normalfont Physics, Math,} BA {\normalfont Philosophy}}
    {Western Washington University, Bellingham WA}
	{}
    %------------------------------------------------
\iffalse
    \entry
    {2013-2015}
    {AS}
    {Whatcom Community College, Bellingham WA}
\fi
    %------------------------------------------------

\end{entrylist}

\vspace{-0.5cm}
%----------------------------------------------------------------------------------------
%	ABOUT
%----------------------------------------------------------------------------------------
%\vspace{-0.8cm}
%\iffalse

%\aside{About Me}{2.035}{
%    \footnotesize
%    I am a third-year PhD Student in the Department of Astronomy at Columbia.
%    My research interests span most areas of computational astrophysics and cosmology.
%    Specifically, I am interested in questions involving supermassive black holes, the early universe, and Population III stars.
%}
%\fi
%----------------------------------------------------------------------------------------
%	POSITIONS
%----------------------------------------------------------------------------------------
\section{Positions}

\begin{entrylist}
    \entry
    {2021--}
    {Graduate Researcher}
    {Columbia University, New York, NY}
    {%
        \vspace{-1em}
        \begin{list}{{\color{numcolor}$-$}}{\cvlist}
	  \item Supermassive star formation and their role in seeding supermassive black holes
          \item Learning the Universe: Using machine learning to accelerate forward modeling of cosmological simulations
          \item Observable consequences of the heavy seed origin for supermassive black holes
          \item SETI: Numerical investigations of star-lifting, identifying observable features of star-lifting

          \item Advised by Zoltan Haiman, Greg Bryan, David Kipping
        \end{list}
    }
    %------------------------------------------------


  \entry
  {2021-2023}
  {Graduate Teaching Assistant}
  {Columbia University, New York, NY}
  {%
  }

    %------------------------------------------------
    \entry
    {2015--2021}
    {Undergraduate Researcher}
    {Western Washington University, Bellingham, WA}
    {%
    }

\end{entrylist}


%----------------------------------------------------------------------------------------
%	AWARDS
%----------------------------------------------------------------------------------------
%\iffalse
\vspace{-0.5cm}
\section{Honors \& Awards}

\begin{entrylist}

    \entry
    {2023-2024}
	{Explore Computing Time: {\normalfont 400,000 CU}}
    {ACCESS}
    {%
        \vspace*{-1.1em}
    }

    \entry
    {2022-2023}
    {Edith and Robert Fehr Fellowship}
    {Columbia U.}
    {%
        \vspace*{-1.1em}
    }

    %------------------------------------------------
    \entry
    {2020}
    {Magna Cum Laude in both BS \& BA}
    {WWU}
    {%
        \vspace*{-1.1em}
    }

    %------------------------------------------------
    \entry
    {2019}
    {Material Science Undergraduate Research Grant}
    {WWU}
    {%
        \vspace*{-1.1em}
    }

    %------------------------------------------------
    \entry
    {2018-2019}
    {Oscar Edwin Olson Scholarship (x2)} % Award
    {WWU}
    {%
        \vspace*{-1.1em}
    }

    %------------------------------------------------
    \entry
    {2018}
    {Willard A. and Anne W. Brown Astronomy Scholarship} % Award
    {WWU}
    {%
        \vspace*{-1.1em}
    }

    %------------------------------------------------
    \entry
    {2018}
    {Summer Student Research Stipend} % Award
    {WWU}
    {%
        \vspace*{-1.1em}
    }

\end{entrylist}
%\fi
%----------------------------------------------------------------------------------------
%	REFS
%----------------------------------------------------------------------------------------
\iffalse
\aside{References}{0.0325}{
    %\textbf{Kevin Covey}\\
    %{\footnotesize waiting@wwu.edu}\\[0.3em]
    %\textbf{Armin Rahmani}\\
    %{\footnotesize dhogg@flatironinstitute.org}\\[0.3em]
    %\textbf{rory barnes}\\
    %{\footnotesize rory@astro.washington.edu}
}
\fi
%----------------------------------------------------------------------------------------
%	METRICS
%----------------------------------------------------------------------------------------
%\iffalse

%\ifdefined \onepage \else
%    \section{metrics}
%
    % SUPER HACKY
  %  \raisebox{-0.9\height}[0pt][0pt]{
  %      \includegraphics[height=1.25in]{metrics.pdf}
  %  }
  %  \pagebreak
%\fi
%\fi
%----------------------------------------------------------------------------------------
%	TEACHING
%----------------------------------------------------------------------------------------
%\ifdefined \onepage \else
%\aside{links}{0.05}{
%    {\color{numcolor}\faVideoCamera} \, \link{https://tinyurl.com/wnxmxb43}{LSST Lecture I}\\[0.5em]
%    {\color{numcolor}\faCloudDownload} \, \link{https://tinyurl.com/9247pj4}{LSST Worksheet I}\\[0.5em]
%    {\color{numcolor}\faVideoCamera} \, \link{https://drive.google.com/file/d/1PW-5Tkwnai7uQAZB4COFBUVTgWMZkE1P/view?usp=sharing}{LSST Lecture II}\\[0.5em]
%    {\color{numcolor}\faCloudDownload} \, \link{https://github.com/LSSTC-DSFP/LSSTC-DSFP-Sessions/tree/main/Session13/Day2}{LSST Worksheet II}\\[0.5em]
%}
%\fi

%\iffalse

\vspace{-0.4cm}
\section{Software}

\begin{entrylist}
    %------------------------------------------------

    \entry
    {\textbf{star\_lifting}}
    {\link{https://github.com/mscoggs/star_lifting}{Repository}}
    {}
    {%
        \vspace{-1em}
        \begin{list}{{\color{numcolor}$-$}}{\cvlist}
            \item A MESA wrapper which evolves the star with a time-depedent mass loss rate, keeping flux on a habitable planet constant.
        \end{list}
    }
    %------------------------------------------------

    \entry
    {\textbf{qubit\_simulation}}
    {\link{https://github.com/mscoggs/qubit_simulation}{Repository}}
    {}
    {%
        \vspace{-1em}
        \begin{list}{{\color{numcolor}$-$}}{\cvlist}
            \item Simulating the evolution of a superconducting chip with the goal of finding patterns in the optimal protocols (values of the controls over time which evolve an initial state into a target state in the shortest possible time) over a variety of initial and target combinations.
        \end{list}
    }

    %------------------------------------------------

    \entry
    {\textbf{no\_wave\_qm}}
    {\link{https://github.com/mscoggs/no_wave_qm}{Repository}}
    {}
    {%
        \vspace{-1em}
        \begin{list}{{\color{numcolor}$-$}}{\cvlist}
            \item Simulating the evolution according to a hamilton-jacobi formulation of QM which replaces the wave with a configuration space density and equations of motion. Trajectory tracking using a 4-th order Runge Kutta technique.
        \end{list}
    }



\end{entrylist}
%\fi
%----------------------------------------------------------------------------------------
%	STUDENTS
%----------------------------------------------------------------------------------------
\iffalse
\vspace{-0.5cm}



  \section{Outreach \& Press}
      \begin{entrylist}

          %------------------------------------------------

          \entry
          {2022}
          {\link{https://www.youtube.com/watch?v=IY0KWLanlLM&ab_channel=CoolWorlds}{Lazarus Stars - Extending Stellar Lifespans by Billions of Years}}
          {Youtube}

          %------------------------------------------------
          \entry
          {2023}
	      {\link{https://www.skyatnightmagazine.com/space-science/reducing-sun-mass-save-earth-in-the-future}{Could reducing the Sun's mass stop it destroying Earth in the future?}}
          {BBC Sky at Night}


          \entry
          {2023}
	      {\link{https://www.inverse.com/science/dyson-swarms-could-keep-aging-stars-from-burning-out}{Aliens Could Build Massive Megastructures to Save Dying Stars}}
          {Inverse}

          \entry
          {2022}
          {\link{https://github.com/mscoggs/intro_to_computational_astro}{Introduction to Computational Astronomy}}
          {Online}

      \end{entrylist}
      %------------------------------------------------



\vspace{-0.5cm}
\fi
%\iffalse
\section{Teaching \& Service}
\begin{entrylist}
\vspace{-0.5cm}


    \entry
    {2023}
    {Journal Reviewer: {\normalfont ApJ, A\&A}}
    {}
    {}
    %------------------------------------------------
    \entry
    {2023-}
	{Associate Director: {\normalfont Student Training in Astronomy Research (STARs)}}
    {Columbia University}
	{}
    %------------------------------------------------

    \entry
    {2021-2023}
    {Graduate Teaching Assistant}
    {Columbia University}
	{}

    %------------------------------------------------

    \entry
    {2020-2021}
    {Mathematics Teaching Assistant}
    {WWU}
	{}

    %------------------------------------------------

    \entry
    {2017-2020}
    {Physics Teaching Assistant}
    {WWU}
    {%
    }

    %------------------------------------------------

    \entry
    {2019}
    {Student Faculty Hiring Committee}
    {WWU}
	{}

    %------------------------------------------------

    \entry
    {2018-2019}
    {Physics Study Group Facilitator}
    {WWU}
    {%
    }

    %------------------------------------------------

    \entry
    {2018-2019}
    {Math Tutoring Fellow}
    {WWU}
    {%
    }


\end{entrylist}
%\fi
%STUDENTS


\vspace{-0.3cm}

%\iffalse
  \section{Mentoring}
      \begin{entrylist}
      %------------------------------------------------
  \entry
  {2024-}
    {STARs program}
    {}
  {%
\vspace{-1em}
\begin{list}{{\color{numcolor}$-$}}{\cvlist}
\item Co-advising with another Graduate student, Daniel Yahalomi
\item Three highschool students are using NASA's Meteoroid Engineering Model (MEM) code to measure
 the impact rates of debris on the lunar surface,
 with the goal of determining the shielding needed for the Artemis mission.

\end{list}
  }
  %------------------------------------------------

	    \entry
	    {2023}
	      {Undergraduate Students}
	      {}
	    {%
		\vspace{-1em}
		\begin{list}{{\color{numcolor}$-$}}{\cvlist}
		\item Andrea Dubbels - Abnormal Photometry in the GAIA DR3 Catalog

		\end{list}
	    }
          %------------------------------------------------
	    \entry
	    {2023-2025}
	      {High School Students}
	      {}
	    {%
		\vspace{-1em}
		\begin{list}{{\color{numcolor}$-$}}{\cvlist}
		\item Students took part in 2-12 month projects designed to expose them to research, typically within astronomy. Some projects have been (or will be) submitted to high school journals.
		    \item Junhao Lei (early acceptance to Cornell) - A review of dark matter (accepted, International Journal of High School Research)
		    \item Iulia Achim - Exploring the potential for habitability around a black hole (under review, Journal of Emerging Investigators)
        \item Jai Nair - Searching for biosignatures in nearby exoplanets (in prep)
        \item Estefania Olaiz - A new triple star system.
		    \item Pratham Aggarwal - The origins of supermassive black holes
		    \item Jiarui Shi, Hiep Duc Nguyen, Weibo Qin, Elenes Diana, William Li

		\end{list}
	    }
          %------------------------------------------------


      \end{entrylist}
      %------------------------------------------------

      \vspace{-1cm}


%\fi

\iffalse
    \aside{}{4.2}{
    {\footnotesize primary developer}\\[1em]
    \hrule
    {\footnotesize secondary developer}
    }
\fi

\iffalse
    \section{Computing Experience}

    \begin{entrylist}

        %------------------------------------------------

        \entry
        {\textbf{Languages}(years): }
        {\textnormal{  \ C++(4), Python(4), C(3), Java(1), Matlab(1), Wolfram(1), Scheme(0.5), SQL(0.5)}}
        {}
        {
        }

        %------------------------------------------------
        \entry
        {\textbf{OS}: }
        {\textnormal{ \ Linux, mostly Ubuntu (5),  \   Windows (10+) \  Mac OS X (1)}}
        {}
        {
        }

        %------------------------------------------------
        \entry
        {\textbf{HPC Experience}: }
        {\textnormal{ WWU's CSCI Cluster \& CSE Cluster (Over 250 CPU years), Stampede2 (ongoing}}
        {}
        {
        }


    \end{entrylist}
\fi



%----------------------------------------------------------------------------------------
%	STATS
%----------------------------------------------------------------------------------------
%\iffalse

%\vspace{0.5cm}
%\aside{stats}{0.075}{
%    \vspace*{0.5em}
%    \input{pubs_summary}
%}
%\fi


%----------------------------------------------------------------------------------------
%	PUBLICATIONS
%----------------------------------------------------------------------------------------


\ifdefined \withpubs
    %
    \ifdefined \citationskip
    \else
        \def\citationskip{0.95}
    \fi
    %\aside{}{\citationskip}{
    %{\footnotesize citations $\longrightarrow$}\\[0em]
    %{\footnotesize (refereed in \textbf{bold})}
    %}
    %
    \section{Publications}
    %
    \begin{list}{}{\pubslist}
        \input{pubs}
    \end{list}
    %
    \vspace{1em}
\fi

%----------------------------------------------------------------------------------------
%	TALKS
%----------------------------------------------------------------------------------------

%\vspace{-0.5cm}

%\ifdefined \withtalks

%    \section{Selected Talks}
    %
%    \vspace{-0.3cm}

%    \begin{list}{}{\pubslist}
%        \input{talks}
%    \end{list}
%\fi

\end{document}
